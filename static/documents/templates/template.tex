% template.tex
% CSCI 0220 course staff, Brown University
% (c) 2021
%%%%%%%%%%%%%%%%%%%%%%%%%%%%%%%%%%%%%%%%%%%%%%%%%%%%%%%%%%%%%%%%%%%%%%%%%%%%%%%%
%%%%%%%%%% PREAMBLE %%%%%%%%%%%%%%%%%%%%%%%%%%%%%%%%%%%%%%%%%%%%%%%%%%%%%%%%%%%%
%%%%%%%%%%%%%%%%%%%%%%%%%%%%%%%%%%%%%%%%%%%%%%%%%%%%%%%%%%%%%%%%%%%%%%%%%%%%%%%%

% The preamble is everything before the \begin{document}. Here, we define the
% kind of document we are writing, import packages, set properties, and define
% our own commands/environments.

%%%%%%%%%% DOCUMENT CLASS AND PACKAGES %%%%%%%%%%%%%%%%%%%%%%%%%%%%%%%%%%%%%%%%%
\documentclass[12pt,letterpaper]{article}

\usepackage{amsmath}               % Environments, etc. for typesetting math.
\usepackage{amsthm}                % Theorem/proof environments.
\usepackage{amssymb}               % Various symbols.
\usepackage{geometry}              % Easy interface to set page margins.
\usepackage{fancyhdr}              % Custom headers/footers, page styles.
\usepackage[shortlabels]{enumitem} % Stuff for enumerate & itemize environments.

%%%%%%%%%% VARIOUS HOUSEKEEPING %%%%%%%%%%%%%%%%%%%%%%%%%%%%%%%%%%%%%%%%%%%%%%%%
\geometry{margin=3cm}              % Sets margins on all sides.
\pagestyle{fancyplain}             % Sets page style.
\headheight 35pt                   % Sets the height of the header.
\headsep    10pt                   % Sets space between the header and the body.

\setlength{\parindent}{0em}        % Sets paragraph indentation.
\setlength{\parskip}{0.5em}        % Sets space between paragraphs.

\setlist[enumerate]{leftmargin=*}  % Normalize left margin in enumerate.
\setlist[itemize]{leftmargin=*}    % Normalize left margin in itemize.

%%%%%%%%%% ASSIGNMENT INFORMATION %%%%%%%%%%%%%%%%%%%%%%%%%%%%%%%%%%%%%%%%%%%%%%
% Edit these as appropriate.
\newcommand\course{CS 22}          % <-- the course
\newcommand\semester{Spring 2021}  % <-- the current semester
\newcommand\hwnum{1}               % <-- homework number

%%%%%%%%%% HEADER %%%%%%%%%%%%%%%%%%%%%%%%%%%%%%%%%%%%%%%%%%%%%%%%%%%%%%%%%%%%%%
\chead{\textbf{\Large Homework \hwnum}}
\rhead{\today}

%%%%%%%%%% CUSTOM THINGS %%%%%%%%%%%%%%%%%%%%%%%%%%%%%%%%%%%%%%%%%%%%%%%%%%%%%%%

% The answer environment definition.
\newenvironment{answer}[1]{
  \subsubsection*{Problem #1}
}

% Add your own commands!

% Here are some examples that your TAs find helpful.
% Un-comment any or all of them to use. They all are to be used in math mode.

% \newcommand{\N}{\mathbb{N}}         % The set of natural numbers ($4 \in \N$).
% \newcommand{\Z}{\mathbb{Z}}         % The set of integers ($-3 \in \Z$).
% \newcommand\lcm{\operatorname{lcm}} % Least common multiple ($\lcm(1,3) = 3$).
% \newcommand\Pow{\ensuremath{\operatorname{\mathcal{P}}}}
% %%%%%%%%%%%%%%%%%%%%%%%%%%%%%%%%%%%  Power set ($|\Pow(\{1\})| = 2$).
% \newcommand\setbuilder[2]{\ensuremath{\left\{#1\;\middle|\;#2\right\}}}
% %%%%%%%%%%%%%%%%%%%%%%%%%%%%%%%%%%%  Set-builder notation.
% %%%%%%%%%%%%%%%%%%%%%%%%%%%%%%%%%%%  ($A^{c} = \setbuilder{x}{x \not\in A}$)







%%%%%%%%%%%%%%%%%%%%%%%%%%%%%%%%%%%%%%%%%%%%%%%%%%%%%%%%%%%%%%%%%%%%%%%%%%%%%%%%
%%%%%%%%%% DOCUMENT %%%%%%%%%%%%%%%%%%%%%%%%%%%%%%%%%%%%%%%%%%%%%%%%%%%%%%%%%%%%
%%%%%%%%%%%%%%%%%%%%%%%%%%%%%%%%%%%%%%%%%%%%%%%%%%%%%%%%%%%%%%%%%%%%%%%%%%%%%%%%

% The document is everything between \begin{document} and \end{document}.
% It contains everything that will show up in the output pdf file,
% besides the header (which is defined above, and appears on every page).

\begin{document}

\newpage

\begin{center}
    \newcommand{\GradeTableHeader}[1]{\quad\textbf{#1}\quad\mbox{}}
    \renewcommand*{\arraystretch}{1}

    {\huge CS22 Homework}\\

    \Large
    \begin{align*}
        \text{\textbf{HW}}&:
	\text{\underline{\hwnum}} \\
    \end{align*}
\end{center}
\newpage

\begin{answer}{1}
Answer to problem 1 goes here.

\begin{itemize}
    \item More answers.
    \item Blah\dots
    \item Hi there!
\end{itemize}
\end{answer}



\begin{answer}{2}
This is the answer to another question!

This is the second paragraph of another question!

\begin{enumerate}[a.]
    \item Maybe it has\textellipsis{}
    \item \textellipsis{}multiple parts\textellipsis{}
        \begin{proof}
            Because I said so.
        \end{proof}
    \item \textellipsis{}to answer.
\end{enumerate}

\end{answer}



\begin{answer}{3}
Third answer.
\end{answer}



\end{document}
